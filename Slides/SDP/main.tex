\documentclass{beamer}

 \pdfmapfile{+sansmathaccent.map}


\mode<presentation>
{
  \usetheme{Warsaw} % or try Darmstadt, Madrid, Warsaw, Rochester, CambridgeUS, ...
  \usecolortheme{crane} % or try seahorse, beaver, crane, wolverine, ...
  \usefonttheme{serif}  % or try serif, structurebold, ...
  \setbeamertemplate{navigation symbols}{}
  \setbeamertemplate{caption}[numbered]
} 

\renewcommand{\familydefault}{\rmdefault}


\hypersetup{
    colorlinks=true,
    linkcolor=blue,
    filecolor=blue,      
    urlcolor=blue,
}

\usepackage{amsmath}
\usepackage{mathtools}

\DeclareMathOperator*{\argmin}{arg\,min}

\usepackage{subcaption}

\usepackage{tikz}
\tikzset{every picture/.style={line width=0.75pt}} %set default line width to 0.75pt        
%%%%%%%%%%%%%%%%%%%%%%%%%%%%%%%%%%%%%%%%%%%%%%%%%%%


\usepackage{listings}
\usepackage{color}
\definecolor{mygreen}{rgb}{0,0.6,0}
\definecolor{mygray}{rgb}{0.5,0.5,0.5}
\definecolor{mymauve}{rgb}{0.58,0,0.82}

\definecolor{cof}{RGB}{255,187,0}
\definecolor{pur}{RGB}{205,137,0}
\definecolor{greeo}{RGB}{91,173,69}
\definecolor{greet}{RGB}{52,111,72}

\usetikzlibrary{fadings}
\usetikzlibrary{patterns}
\usetikzlibrary{shadows.blur}
\usetikzlibrary{shapes}

\lstset{ 
  backgroundcolor=\color{white},   % choose the background color; you must add \usepackage{color} or \usepackage{xcolor}; should come as last argument
  basicstyle=\footnotesize,        % the size of the fonts that are used for the code
  breakatwhitespace=false,         % sets if automatic breaks should only happen at whitespace
  breaklines=true,                 % sets automatic line breaking
  captionpos=b,                    % sets the caption-position to bottom
  commentstyle=\color{mygreen},    % comment style
  deletekeywords={...},            % if you want to delete keywords from the given language
  escapeinside={\%*}{*)},          % if you want to add LaTeX within your code
  extendedchars=true,              % lets you use non-ASCII characters; for 8-bits encodings only, does not work with UTF-8
  firstnumber=0000,                % start line enumeration with line 0000
  frame=single,	                   % adds a frame around the code
  keepspaces=true,                 % keeps spaces in text, useful for keeping indentation of code (possibly needs columns=flexible)
  keywordstyle=\color{blue},       % keyword style
  language=Octave,                 % the language of the code
  morekeywords={*,...},            % if you want to add more keywords to the set
  numbers=left,                    % where to put the line-numbers; possible values are (none, left, right)
  numbersep=5pt,                   % how far the line-numbers are from the code
  numberstyle=\tiny\color{mygray}, % the style that is used for the line-numbers
  rulecolor=\color{black},         % if not set, the frame-color may be changed on line-breaks within not-black text (e.g. comments (green here))
  showspaces=false,                % show spaces everywhere adding particular underscores; it overrides 'showstringspaces'
  showstringspaces=false,          % underline spaces within strings only
  showtabs=false,                  % show tabs within strings adding particular underscores
  stepnumber=2,                    % the step between two line-numbers. If it's 1, each line will be numbered
  stringstyle=\color{mymauve},     % string literal style
  tabsize=2,	                   % sets default tabsize to 2 spaces
  title=\lstname                   % show the filename of files included with \lstinputlisting; also try caption instead of title
}




%%%%%%%%%%%%%%%%%%%%%%%%%%%%%%%%%%%%%%%%%%%%%%%%%%%%%


\title{Semidefinite Programming}
\subtitle{Computational Intelligence, Lecture 10}
\author{by Sergei Savin}
\centering
\date{Fall 2020}



\begin{document}
\maketitle


\begin{frame}{Content}

\begin{itemize}
\item  Semidefinite Programming (SDP)
\begin{itemize}
    \item General form
    \item Multiple LMI
    \item SDP decision variable
\end{itemize}
\item  Example 1:  Continuous Lyapunov equation as an SDP/LMI
\begin{itemize}
    \item Mathematical formulation
    \item Code
\end{itemize}
\item  Example 2:  Discrete Lyapunov equation as an SDP/LMI
\begin{itemize}
    \item Mathematical formulation
    \item Code
\end{itemize}
\item  Example 3:  LMI Controller design for continuous LTI
\begin{itemize}
    \item Mathematical formulation
    \item Code
\end{itemize}
\item Homework
\end{itemize}

\end{frame}



\begin{frame}{Semidefinite Programming (SDP)}
\framesubtitle{General form}
\begin{flushleft}

General form of a semidefinite program is:

%
\begin{equation}
\begin{aligned}
& \underset{\mathbf{x}}{\text{minimize}}
& & \mathbf{c}^\top\mathbf{x}, \\
& \text{subject to}
& & \begin{cases}
    \mathbf{G} + \sum \mathbf{F}_i x_i \preceq 0, \\
    \mathbf{A}\mathbf{x} = \mathbf{b}.
    \end{cases}
\end{aligned}
\end{equation}

where $\mathbf{F}_i \succeq 0$ and $\mathbf{G} \succeq 0$ (meaning they are positive semidefinite).

\bigskip

Constraint $\mathbf{G} + \sum \mathbf{F}_i x_i \preceq 0$ is called \emph{linear matrix inequality} or \emph{LMI}.
 
\end{flushleft}
\end{frame}




\begin{frame}{Semidefinite Programming (SDP)}
\framesubtitle{Multiple LMI}
\begin{flushleft}

SDP can have several LMIs. Assume you have:

\begin{equation}
    \begin{cases}
        \mathbf{G} + \sum \mathbf{F}_i x_i \preceq 0 \\
        \mathbf{D} + \sum \mathbf{H}_i x_i \preceq 0
    \end{cases}
\end{equation}


This is equivalent to:

\begin{equation}
    \begin{bmatrix} 
            \mathbf{G} & \mathbf{0} \\
            \mathbf{0} & \mathbf{D}
    \end{bmatrix} +
    \sum
    \begin{bmatrix} 
            \mathbf{F}_i & \mathbf{0} \\
            \mathbf{0}   & \mathbf{H}_i
    \end{bmatrix}
    x_i \preceq 0
\end{equation}

\end{flushleft}
\end{frame}




\begin{frame}{Semidefinite Programming (SDP)}
\framesubtitle{SDP decision variable}
\begin{flushleft}

Sometimes it is easier to directly think of semidefinite matrices as of decision variables. This leads to programs with such formulation:
%
\begin{equation}
\begin{aligned}
& \underset{\mathbf{X}}{\text{minimize}}
& & f(\mathbf{X}), \\
& \text{subject to}
& & \begin{cases}
    \mathbf{X} \preceq 0, \\
    \mathbf{g}(\mathbf{X}) = \mathbf{0}.
    \end{cases}
\end{aligned}
\end{equation}

where cost and constraints should adhere to SDP limitations.

\end{flushleft}
\end{frame}



\begin{frame}{Example 1: Continuous Lyapunov equation as an SDP/LMI}
\framesubtitle{Mathematical formulation}
\begin{flushleft}

In control theory, Lyapunov equation is a condition of whether or not a continuous LTI system $\dot{\mathbf{x}} = \mathbf{A}\mathbf{x}$ is stabilizable:

\begin{equation}
    \begin{cases}
        \mathbf{A}^\top\mathbf{P} + \mathbf{P}\mathbf{A} + \mathbf{Q} = 0 \\
        \mathbf{P} \succeq 0 \\
        \mathbf{Q} \succeq 0 
    \end{cases}
\end{equation}
%
where decision variable is $\mathbf{P}$. This can be represented as an SDP:
%
\begin{equation}
\begin{aligned}
& \underset{\mathbf{P}}{\text{minimize}}
& & 0, \\
& \text{subject to}
& & \begin{cases}
    \mathbf{P} \succeq 0, \\
    \mathbf{A}^\top\mathbf{P} + \mathbf{P}\mathbf{A} + \mathbf{Q} = 0.
    \end{cases}
\end{aligned}
\end{equation}


\end{flushleft}
\end{frame}




\begin{frame}{Example 1: Continuous Lyapunov equation as an SDP/LMI}
\framesubtitle{Code}
\begin{flushleft}

\begin{lstlisting}[language=Matlab]
func = @(t) t^2;
derivative_func = @(t) 2*t;

approx_points = [-1, -0.3, 0, 0.3, 1];
n = length(approx_points);
a = zeros(n, 1); 
b = zeros(n, 1);

for i = 1:n
    t = approx_points(i);
    a(i) = derivative_func(t); 
    b(i) = func(t) - a(i)*t ;
end

f = [1; 0];
lin_A = [-ones(n, 1), a];
lin_b = -b;
x = linprog(f, lin_A,lin_b, [], []);

\end{lstlisting}

\end{flushleft}
\end{frame}




\begin{frame}{Example 2: Discrete Lyapunov equation as an SDP/LMI}
\framesubtitle{Mathematical formulation}
\begin{flushleft}

In control theory, Discrete Lyapunov equation is a condition of whether or not a discrete LTI system $\mathbf{x}_{i+1} = \mathbf{A}\mathbf{x}_i$ is stabilizable:

\begin{equation}
    \begin{cases}
        \mathbf{A}^\top \mathbf{P}\mathbf{A} - \mathbf{P} + \mathbf{Q} = 0 \\
        \mathbf{P} \succeq 0 \\
        \mathbf{Q} \succeq 0 
    \end{cases}
\end{equation}
%
where decision variable is $\mathbf{P}$. This can be represented as an SDP:
%
\begin{equation}
\begin{aligned}
& \underset{\mathbf{P}}{\text{minimize}}
& & 0, \\
& \text{subject to}
& & \begin{cases}
    \mathbf{P} \succeq 0, \\
    \mathbf{A}^\top \mathbf{P}\mathbf{A} - \mathbf{P} + \mathbf{Q} = 0.
    \end{cases}
\end{aligned}
\end{equation}


\end{flushleft}
\end{frame}



\begin{frame}{Example 2: Discrete Lyapunov equation as an SDP/LMI}
\framesubtitle{Code}
\begin{flushleft}

\begin{lstlisting}[language=Matlab]
number_of_steps = 7; 
start_point = sum(V1, 2) / size(V1, 2);
finish_point = sum(V3, 2) / size(V3, 2);
weight_goal = 5;
bigM = 15;

cvx_begin
    variable x(n, number_of_steps)
    binary variable c(3, number_of_steps);

    cost = 0;
    for i = 1:(number_of_steps-1)
        cost = cost + norm(x(:, i) - x(:, i+1));
    end
    cost = cost + norm(x(:, 1) - start_point)*weight_goal;
    cost = cost + norm(x(:, number_of_steps) - finish_point)*weight_goal;
    minimize( cost )
\end{lstlisting}


\end{flushleft}
\end{frame}





\begin{frame}{Example 3: LMI Controller design for continuous LTI}
\framesubtitle{Mathematical formulation}
\begin{flushleft}

For an LTI system of the form $\dot{\mathbf{x}} = \mathbf{A}\mathbf{x} + \mathbf{B}\mathbf{u}$ there is an LMI condition to determine if it can be stabilized:

\begin{equation}
    \begin{cases}
        \mathbf{A}\mathbf{P} + \mathbf{P}\mathbf{A}^\top + \mathbf{B}\mathbf{L} + \mathbf{L}\mathbf{B}^\top + \mathbf{Q} = 0 \\
        \mathbf{P} \succeq 0 \\
        \mathbf{Q} \succeq 0 
    \end{cases}
\end{equation}
%
where decision variables are $\mathbf{P}$ and $\mathbf{L}$. 

\bigskip

This gives as a direct way to calculate linear feedback controller $\mathbf{u} = \mathbf{K}\mathbf{x}$ (note the sign!) gains:

\begin{equation}
    \mathbf{K} = \mathbf{L}\mathbf{P}^{-1}
\end{equation}

\end{flushleft}
\end{frame}




\begin{frame}{Example 3: LMI Controller design for continuous LTI}
\framesubtitle{Code}
\begin{flushleft}

\begin{lstlisting}[language=Matlab]
    subject to
        for i = 1:number_of_steps
            A1*x(:, i) <= b1 + (1 - c(1, i))*bigM;
            A2*x(:, i) <= b2 + (1 - c(2, i))*bigM;
            A3*x(:, i) <= b3 + (1 - c(3, i))*bigM;
            c(1, i) + c(2, i) + c(3, i) == 1;
        end
cvx_end

plot(x(1, :)', x(2, :)', '^', 'MarkerEdgeColor', 'k', 'MarkerSize', 10, 'LineWidth', 2); hold on;
\end{lstlisting}

\end{flushleft}
\end{frame}





\begin{frame}{Homework}
% \framesubtitle{Parameter estimation}
\begin{flushleft}


Implement both examples from page 2 of the \href{http://stanford.edu/class/ee363/notes/lmi-cvx.pdf}{LMI CVX documents}.

\end{flushleft}
\end{frame}



\begin{frame}
\centerline{Lecture slides are available via Moodle.}
\bigskip
\centerline{You can help improve these slides at:}

\centerline{\href{https://github.com/SergeiSa/Computational-Intelligence-Slides-Fall-2020}{github.com/SergeiSa/Computational-Intelligence-Slides-Fall-2020}}


\bigskip
\centerline{Check Moodle for additional links, videos, textbook suggestions.}
\end{frame}

\end{document}
