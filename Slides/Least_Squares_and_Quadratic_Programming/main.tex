\documentclass{beamer}

 \pdfmapfile{+sansmathaccent.map}


\mode<presentation>
{
  \usetheme{Warsaw} % or try Darmstadt, Madrid, Warsaw, Rochester, CambridgeUS, ...
  \usecolortheme{crane} % or try seahorse, beaver, crane, wolverine, ...
  \usefonttheme{serif}  % or try serif, structurebold, ...
  \setbeamertemplate{navigation symbols}{}
  \setbeamertemplate{caption}[numbered]
} 

\renewcommand{\familydefault}{\rmdefault}


\hypersetup{
    colorlinks=true,
    linkcolor=blue,
    filecolor=blue,      
    urlcolor=blue,
}

\usepackage{amsmath}
\usepackage{mathtools}

\DeclareMathOperator*{\argmin}{arg\,min}

\usepackage{subcaption}

%%%%%%%%%%%%%%%%%%%%%%%%%%%%%%%%%%%%%%%%%%%%%%%%%%%




\usepackage{listings}
\usepackage{color}
\definecolor{mygreen}{rgb}{0,0.6,0}
\definecolor{mygray}{rgb}{0.5,0.5,0.5}
\definecolor{mymauve}{rgb}{0.58,0,0.82}
\lstset{ 
  backgroundcolor=\color{white},   % choose the background color; you must add \usepackage{color} or \usepackage{xcolor}; should come as last argument
  basicstyle=\footnotesize,        % the size of the fonts that are used for the code
  breakatwhitespace=false,         % sets if automatic breaks should only happen at whitespace
  breaklines=true,                 % sets automatic line breaking
  captionpos=b,                    % sets the caption-position to bottom
  commentstyle=\color{mygreen},    % comment style
  deletekeywords={...},            % if you want to delete keywords from the given language
  escapeinside={\%*}{*)},          % if you want to add LaTeX within your code
  extendedchars=true,              % lets you use non-ASCII characters; for 8-bits encodings only, does not work with UTF-8
  firstnumber=0000,                % start line enumeration with line 0000
  frame=single,	                   % adds a frame around the code
  keepspaces=true,                 % keeps spaces in text, useful for keeping indentation of code (possibly needs columns=flexible)
  keywordstyle=\color{blue},       % keyword style
  language=Octave,                 % the language of the code
  morekeywords={*,...},            % if you want to add more keywords to the set
  numbers=left,                    % where to put the line-numbers; possible values are (none, left, right)
  numbersep=5pt,                   % how far the line-numbers are from the code
  numberstyle=\tiny\color{mygray}, % the style that is used for the line-numbers
  rulecolor=\color{black},         % if not set, the frame-color may be changed on line-breaks within not-black text (e.g. comments (green here))
  showspaces=false,                % show spaces everywhere adding particular underscores; it overrides 'showstringspaces'
  showstringspaces=false,          % underline spaces within strings only
  showtabs=false,                  % show tabs within strings adding particular underscores
  stepnumber=2,                    % the step between two line-numbers. If it's 1, each line will be numbered
  stringstyle=\color{mymauve},     % string literal style
  tabsize=2,	                   % sets default tabsize to 2 spaces
  title=\lstname                   % show the filename of files included with \lstinputlisting; also try caption instead of title
}




%%%%%%%%%%%%%%%%%%%%%%%%%%%%%%%%%%%%%%%%%%%%%%%%%%%%%


\title{Least Squares and Quadratic Programming}
\subtitle{Computational Intelligence, Lecture 5}
\author{by Sergei Savin}
\centering
\date{Fall 2020}



\begin{document}
\maketitle


\begin{frame}{Content}

\begin{itemize}
\item Some problems with exact analytical solutions
\item Problems with inequality constraints
\item Quadratic programming
\item Homework
\end{itemize}

\end{frame}



\begin{frame}{Some problems with exact analytical solutions}
% \framesubtitle{Parameter estimation}
\begin{flushleft}

Problem 1. $\text{minimize} \ ||\mathbf{x}||$. Solution $\mathbf{x} = \mathbf{0}$.

\bigskip

Problem 2. $\text{minimize} \ ||\mathbf{A}\mathbf{x}||$. Solution $\mathbf{x} = \mathbf{0}$.

\bigskip

Problem 3. $\text{minimize} \ ||\mathbf{A}\mathbf{x} + \mathbf{b}||$. 

\bigskip

We observe that minimum of $||\mathbf{A}\mathbf{x} + \mathbf{b}||$ coincides with the minimum of $(\mathbf{A}\mathbf{x} + \mathbf{b})^2 = (\mathbf{A}\mathbf{x} + \mathbf{b})^\top (\mathbf{A}\mathbf{x} + \mathbf{b}) = \mathbf{b}^\top\mathbf{b} + \mathbf{b}^\top\mathbf{A}\mathbf{x} + \mathbf{x}^\top\mathbf{A}^\top\mathbf{b} + \mathbf{x}^\top\mathbf{A}^\top\mathbf{A}\mathbf{x}$, whose minimum coincides with the minimum of $\mathbf{b}^\top\mathbf{A}\mathbf{x} + \mathbf{x}^\top\mathbf{A}^\top\mathbf{b} + \mathbf{x}^\top\mathbf{A}^\top\mathbf{A}\mathbf{x}$. Since $\mathbf{b}^\top\mathbf{A}\mathbf{x}$ is a scalar, it is equal to its own transpose $(\mathbf{b}^\top\mathbf{A}\mathbf{x})^\top = \mathbf{x}^\top\mathbf{A}^\top\mathbf{b}$.

Taking derivative with respect to $\mathbf{x}$ we find gradient of the obtained \emph{cost function}, and set it to zero, since that is the condition for the extreme point:

\begin{equation}
    \frac{\partial}{\partial \mathbf{x}} (2\mathbf{b}^\top\mathbf{A}\mathbf{x} + \mathbf{x}^\top\mathbf{A}^\top\mathbf{A}\mathbf{x}) = 2\mathbf{b}^\top\mathbf{A} + 2\mathbf{x}^\top\mathbf{A}^\top\mathbf{A} = \mathbf{0}
\end{equation}
 
\end{flushleft}
\end{frame}

\begin{frame}{Some problems with exact analytical solutions}
\framesubtitle{Part 2}
\begin{flushleft}

Thus we get expression: $2\mathbf{x}^\top\mathbf{A}^\top\mathbf{A} = -2\mathbf{b}^\top\mathbf{A}$, which we transpose and find value of $\mathbf{x}$:

\begin{equation} \label{pinv}
\mathbf{x} = -(\mathbf{A}^\top\mathbf{A})^{-1}\mathbf{A}^\top\mathbf{b}
\end{equation}

The shorthand for this formula is $\mathbf{x} = -\mathbf{A}^+\mathbf{b}$, and it is called Moore-Penrose pseudoinverse.

\bigskip

Remember that a projector is often defined as $\mathbf{P} = \mathbf{A}\mathbf{A}^+$. Now we can provide an explicit formula for a projector:

\begin{equation} \label{projector}
\mathbf{P} = \mathbf{A}(\mathbf{A}^\top\mathbf{A})^{-1}\mathbf{A}^\top
\end{equation}

However, neither \eqref{pinv} not \eqref{projector} should be directly used in as algorithms, since usually there are better ways to compute those quantities (via SVD decomposition, for example).

\end{flushleft}
\end{frame}





\begin{frame}{Some problems with exact analytical solutions}
\framesubtitle{Part 3}
\begin{flushleft}

Problem 4. 
%
\begin{equation}
\begin{aligned}
& \underset{\mathbf{x}}{\text{minimize}}
& & || \mathbf{x} ||, \\
& \text{subject to}
& & \mathbf{A} \mathbf{x} + \mathbf{B} \mathbf{y} = \mathbf{c}.
\end{aligned}
\end{equation}

We already solved this problem before, during lecture 2, using projection. Solution is: $\mathbf{x} = -((\mathbf{I} - \mathbf{B}\mathbf{B}^+) \mathbf{A})^+(\mathbf{I} - \mathbf{B}\mathbf{B}^+) \mathbf{c}$.

\bigskip

Problem 5. 
%
\begin{equation}
\begin{aligned}
& \underset{\mathbf{x}}{\text{minimize}}
& & || \mathbf{x} ||, \\
& \text{subject to}
& & \mathbf{A} \mathbf{x} = \mathbf{c}.
\end{aligned}
\end{equation}

We can treat is as a special case of the previous problem, where $\mathbf{B} = \mathbf{0}$, hence $(\mathbf{I} - \mathbf{B}\mathbf{B}^+) = \mathbf{I}$, and the solution is: $\mathbf{x} = -\mathbf{A}^+\mathbf{c}$.

\bigskip

Notice that solution for the problem 5 and problem 3 are written identically, even though problem 3 asks us to minimize residual of the linear system, while problem 5 - find minimum norm solution.

\end{flushleft}
\end{frame}





\begin{frame}{Some problems with exact analytical solutions}
\framesubtitle{Part 4}
\begin{flushleft}

Problem 6. 
%
\begin{equation}
\begin{aligned}
& \underset{\mathbf{x}}{\text{minimize}}
& & || \mathbf{D}\mathbf{x} ||, \\
& \text{subject to}
& & \mathbf{A} \mathbf{x} = \mathbf{b}.
\end{aligned}
\end{equation}

One way to think about it is to first find all solution to the constraint equation $\mathbf{A} \mathbf{x} = \mathbf{b}$ and then find optimal one among them. As we know, all solutions are given as: $\mathbf{x} = \mathbf{A}^+\mathbf{b} + \mathbf{N}\mathbf{z}$, where $\mathbf{N} = \text{null}(\mathbf{A})$. Then our cost function becomes: $|| \mathbf{D}\mathbf{A}^+\mathbf{b} +  \mathbf{D}\mathbf{N}\mathbf{z} ||$, which is equivalent to the problem 3. Thus, we can write solution as: $\mathbf{z}^* = -(\mathbf{D}\mathbf{N})^+ \mathbf{D}\mathbf{A}^+\mathbf{b}$. In terms of $\mathbf{x}$ solution is:

\begin{equation}
    \mathbf{x}^* = \mathbf{A}^+\mathbf{b}-\mathbf{N}(\mathbf{D}\mathbf{N})^+ \mathbf{D}\mathbf{A}^+\mathbf{b}
\end{equation}

\end{flushleft}
\end{frame}



\begin{frame}{Some problems with exact analytical solutions}
\framesubtitle{Part 5}
\begin{flushleft}

Problem 7. 
%
\begin{equation}
\begin{aligned}
& \underset{\mathbf{x}}{\text{minimize}}
& & || \mathbf{D}\mathbf{x} + \mathbf{f} ||, \\
& \text{subject to}
& & \mathbf{A} \mathbf{x} = \mathbf{b}.
\end{aligned}
\end{equation}

After the same initial step, we arrive at the cost function $|| \mathbf{D}\mathbf{N}\mathbf{z} + \mathbf{D}\mathbf{A}^+\mathbf{b} + \mathbf{f}||$. It is only different in the constant term, and the solution is found as follows:


\begin{equation}
    \mathbf{z}^* = -(\mathbf{D}\mathbf{N})^+ (\mathbf{D}\mathbf{A}^+\mathbf{b} + \mathbf{f})
\end{equation}
\begin{equation}
    \mathbf{x}^* = \mathbf{A}^+\mathbf{b}-\mathbf{N}(\mathbf{D}\mathbf{N})^+ (\mathbf{D}\mathbf{A}^+\mathbf{b} + \mathbf{f})
\end{equation}


\end{flushleft}
\end{frame}



\begin{frame}{Some problems with exact analytical solutions}
\framesubtitle{Part 6}
\begin{flushleft}

Problem 8. 
%
\begin{equation}
\begin{aligned}
& \underset{\mathbf{x}}{\text{minimize}}
& & \mathbf{x}^\top \mathbf{H} \mathbf{x} + \mathbf{c}^\top\mathbf{x}, \\
& \text{subject to}
& & \mathbf{A} \mathbf{x} = \mathbf{b}.
\end{aligned}
\end{equation}

where $\mathbf{H}$ is positive-definite.

\bigskip

Assume that we found a decomposition $\mathbf{H} = \mathbf{D}^\top\mathbf{D}$. We can also find such $\mathbf{f}$ that $2\mathbf{f}^\top\mathbf{D} = \mathbf{c}^\top$. Then our cost function becomes $\mathbf{x}^\top \mathbf{D}^\top\mathbf{D} \mathbf{x} + 2\mathbf{f}^\top\mathbf{D}\mathbf{x}$, which as we saw before has coinciding minimum with the cost function $||\mathbf{D}\mathbf{x} + \mathbf{f}||$.

\bigskip

Therefore the problem has the same solution as Problem 6, after the mentioned above change in constants.

\end{flushleft}
\end{frame}




\begin{frame}{Problems with inequality constraints}
% \framesubtitle{Part 1}
\begin{flushleft}

Problem 9. 
%
\begin{equation}
\begin{aligned}
& \underset{\mathbf{x}}{\text{minimize}}
& & || \mathbf{D}\mathbf{x} + \mathbf{f} ||, \\
& \text{subject to}
& & \mathbf{x} \leq \mathbf{b}.
\end{aligned}
\end{equation}

Problem 10. 
%
\begin{equation}
\begin{aligned}
& \underset{\mathbf{x}}{\text{minimize}}
& & || \mathbf{x} ||, \\
& \text{subject to}
& & \mathbf{A}\mathbf{x} \leq \mathbf{b}.
\end{aligned}
\end{equation}

Problem 11. 
%
\begin{equation}
\begin{aligned}
& \underset{\mathbf{x}}{\text{minimize}}
& & || \mathbf{D}\mathbf{x} + \mathbf{f} ||, \\
& \text{subject to}
& & \begin{cases}
    \mathbf{A}\mathbf{x} \leq \mathbf{b}, \\
    \mathbf{C}\mathbf{x} = \mathbf{d}.
    \end{cases}
\end{aligned}
\end{equation}

\end{flushleft}
\end{frame}



\begin{frame}{Quadratic programming}
% \framesubtitle{Part 1}
\begin{flushleft}

Mentioned problems can be described together as quadratic programs. The name is due to the cost function being quadratic (or equivalent). They are allowed to have linear equality or inequality constraints. 

\bigskip

General form of a quadratic program is given below:

%
\begin{equation}
\begin{aligned}
& \underset{\mathbf{x}}{\text{minimize}}
& & \mathbf{x}^\top \mathbf{H} \mathbf{x} + \mathbf{f}^\top\mathbf{x}, \\
& \text{subject to}
& & \begin{cases}
    \mathbf{A}\mathbf{x} \leq \mathbf{b}, \\
    \mathbf{C}\mathbf{x} = \mathbf{d}.
    \end{cases}
\end{aligned}
\end{equation}

where $\mathbf{H}$ is positive-definite and $\mathbf{A}\mathbf{x} \leq \mathbf{b}$ describe a \emph{convex region}.

\end{flushleft}
\end{frame}



\begin{frame}{Homework}
% \framesubtitle{Parameter estimation}
\begin{flushleft}

\begin{itemize}
    \item Solve all examples of problems with and without inequalities using \texttt{quadprog} function in the language of your choice.
\end{itemize}

\end{flushleft}
\end{frame}




\begin{frame}
\centerline{Lecture slides are available via Moodle.}
\bigskip
\centerline{You can help improve these slides at:}

\centerline{\href{https://github.com/SergeiSa/Computational-Intelligence-Slides-Fall-2020}{github.com/SergeiSa/Computational-Intelligence-Slides-Fall-2020}}


\bigskip
\centerline{Check Moodle for additional links, videos, textbook suggestions.}
\end{frame}

\end{document}
